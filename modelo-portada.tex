% === No modificar éstos parámetros === 

\documentclass[letterpaper, 12pt]{report}
\usepackage[utf8]{inputenc}
\usepackage[english, spanish]{babel}
\usepackage{fullpage} % changes the margin
\usepackage{graphicx} 
\usepackage{enumitem} 
\usepackage{chngcntr}
\counterwithin{figure}{section}
\renewcommand{\thesection}{\arabic{section}} 
\renewcommand{\thesubsection}{\thesection.\arabic{subsection}}
\renewcommand{\baselinestretch}{1.5}


\begin{document}
% Hoja de portada, únicamente editar nombres y códigos

\begin{titlepage}
	\centering
	%\includegraphics[width=0.3\textwidth]{logo_utb.png}\par\vspace{1cm}
	%{\scshape\LARGE Universidad Tecnol\'ogica de Bol\'ivar \par}
	%\vspace{1cm}
	{\scshape\Large Universidad Central de Venezuela \par}
    {\scshape\Large Facultad de ingenier\'ia \par}
    {\scshape\Large Escuela de ingenier\'ia El\'ectrica \par}
    {\scshape\Large Departamento de Electr\'onica, Computaci\'on y Control \par}
	%\vspace{.2cm}
    \vspace{4cm}
    %{\scshape\Large IETR1433 \par}
	%\vspace{1cm}
	{\Large\bfseries LAB 2 - Rectificadores\par}
	\vspace{4cm}
	{\itshape Estudiante 1, 29571461 \par}
    {\itshape Estudiante 2, T000XXXX \par}
    %{\itshape Estudiante 3, T000XXXX \par}
    %{\itshape Estudiante 4, T000XXXX \par}
	\vfill
	Elaborado por\par
	Ricardo Santana\par
    Carla Fajardo
	\vfill
	{\large \today\par}
\end{titlepage}

\begin{abstract}
In this lab, the student will learn about the operation of a single-phase diode rectifier with a resistive load and their performance parameters. The student will verify the impact of a half-wave and a full-wave rectified sine wave on the power quality of the networks it is attached to, by studying its harmonic distortion and power factor. 
\end{abstract}

\section{Materiales}

\subsection{Mult\'imetro}

En esta secci\'on se introducen las caracter\'isticas t\'ecnicas de el equipo de medici\'on empleado. Se deben incluir los valores m\'aximos de medicion al igual que la precisi\'on de la medici\'on y tolerancias. 

\subsection{Re\'ostato}

En esta secci\'on se debe suministrar la informaci\'on de las caracter\'isticas t\'ecnicas de el equipo empleado. Se deben incluir el valor nominal y medido del re\'ostato. 

\subsection{Diodo de Potencia}

Las especificaciones del fabricante al igual que las condiciones m\'aximas de operaci\'on determinaddas con el dise\~no del disipador.  

\subsection{Pinza Amperim\'etrica}

En esta secci\'on se introducen las caracter\'isticas t\'ecnicas de el equipo de medici\'on empleado. Se deben incluir los valores m\'aximos de medicion al igual que la precisi\'on de la medici\'on y tolerancias. 

\newpage

\section{Metodolog\'ia}

En \'esta secci\'on se debe realizar una descripci\'on breve del procedimiento realizado. Puede ser separado en fases 1, 2, 3...n. 

\newpage

\section{An\'alisis de Resultados}
Esta es la secci\'on m\'as importante del informe. Aqu\'i se deben registrar todas las mediciones obtenidas durante las pr\'acticas de laboratorio. Se deben incluir las comparaciones con los datos te\'oricos y los obtenidos experimentales. Todos los c\'alculos de eficiencia deben ser registrados aqu\'i. 

\subsection{Resultados Experimentales}
Las mediciones realizadas deben ser ingresadas en \'esta subsecci\'on. Se deben incluir los resultados simulados y ser comparados con los datos experimentales. Por ejemplo, para el caso de un circuito rectificador: 

\begin{itemize}[leftmargin=1.8cm]
  \item Forma de onda de voltaje
  \item Forma de onda de corriente
  \item Valor DC de voltaje y corriente en la carga
  \item Valor RMS de voltaje y corriente en la carga
  \item Factor de utilizaci\'on del transformador
  \item Distorsi\'on harm\'onica de corriente
  \item Distorsi\'on harm\'onica de voltaje
\end{itemize} 

\begin{figure}
\centering
\includegraphics[width=0.5\linewidth]{figure1.png}
	\label{fig:1}
	\caption{Montaje del Circuito de Laboratorio}
\end{figure}

\subsection{An\'alisis de Eficiencia}
Todos los c\'alculos de eficiencia de conversi\'n deben ser ingresados en \'esta secci\'on. Por ejemplo, para un circuito rectificador: 

\begin{itemize}[leftmargin=1.8cm]
  \item Efectividad de Rectificaci\'on
  \item Factor de forma
  \item Factor de rizado
  \item Factor de potencia
  \item Eficiencia de Conversi\'on $(P_{dc}/P_{ac})$
\end{itemize}

Las t\'ablas pueden ser insertadas como im\'agenes o de forma nativa \LaTeX\ as\'i: 

\begin{center}
\begin{tabular}{| l | c |}
 \hline
 \textbf{Elemento} & \textbf{Cantidad}  \\ 
 \hline
 Diodo de Potencia & 1  \\  
 Re\'ostato 33 $\Omega$ & 3 \\
 Banco de Potencia & 1 \\
 Mult\'imetro & 1 \\
 Pinza amper\'imetrica & 1 \\
 Cables de Conexi\'on & 5 (min) \\
 \hline
\end{tabular}
\end{center}

\newpage

\section{Conclusiones y Aplicaciones}
En \'esta secci\'on se deben explicar en detalle los aspectos m\'as relevantes de la pr\'actica realizada, aprendizaje obtenido y posibles usos y aplicaciones del montaje realizado. 

\begin{thebibliography}{9}
\bibitem{rashid} Rashid, M. H. (2001). Power Electronics Handbook. Power Electronics Handbook (Third Edition).http://dx.doi.org/10.1016/B978-0-12-382036-5.00036-7
\end{thebibliography}

\end{document}
Quiénes somos
About us
Our values